\documentclass[letterpaper,8pt]{article}

\usepackage[letterpaper,margin=0.75in]{geometry}

\usepackage{nicefrac}
\usepackage{pstricks}
\usepackage{pst-plot}
\usepackage{pst-node}
\usepackage{pst-circ}
\usepackage{pst-coil}
\usepackage{multido}
\usepackage{booktabs}
\usepackage{multirow}
\usepackage{amsmath,mathtools,nicefrac}
\usepackage{soul}
\usepackage{psvectorian}


%\renewcommand{\familydefault}{\sfdefault}
\usepackage{helvet}
\usepackage{mathpazo}

\setlength{\parindent}{0pt}
\setlength{\parskip}{2.5ex}
\usepackage{setspace}
\usepackage[superscript,biblabel]{cite}

% disables chapter, section and subsection numbering
\setcounter{secnumdepth}{-1} 


\begin{document}

\fbox{\parbox{\linewidth}{\em \LARGE \centering Pendulum Problems }}

\section{Lagrange's Equation}

Basically it is the `next level' of Newton's laws of motion.  It allows one to solve much more
complicated dynamics systems.  

The Lagranian is defined as:
\[
\mathsf{Lagrangian} = \mathsf{Kinetic Energy} - \mathsf{Potential Energy} \Rightarrow \mathcal{L} = T - V
\]

To obtain the equations of motion, we need to solve:
\[
\frac{d}{dt}\left( \frac{\partial\mathcal{L}}{\partial \dot{q}_i} \right) - 
\frac{\partial\mathcal{L}}{\partial q_i} = 0
\]


\section{Simple Pendulum}

This is perhaps the simpler cases, a pendulum without damping or any external forces...
\begin{center}
\psset{unit=1in}
\begin{pspicture}(0,0.25)(4,2)
\psframe[fillstyle=hlines,hatchcolor=gray,hatchsep=1pt](1.5,2)(2.5,1.9)
\psline(2,1.9)(3,0.5)
\pscircle[fillstyle=solid,fillcolor=black](2,1.9){1.5pt}
\pscircle[fillstyle=solid,fillcolor=gray](3,0.5){3pt}
\rput(2.5,1.35){$\ell$}
\rput(3.15,0.5){$m$}
\psline[linewidth=0pt](2,1.75)(2,0.75)
\rput(2.35,0.85){$\theta$}
\psarc[linewidth=0.5pt]{<-}(2,1.9){1.1}{270}{285}
\psarc[linewidth=0.5pt]{->}(2,1.9){1.1}{292}{305.6}
\end{pspicture}
\end{center}

If the origin is at the location of where the pendulum is fixed to the ceiling, then the location of the mass 
can be calculated as:
\[
x = \ell \sin \theta 
\qquad\qquad
y = -\ell \cos \theta
\]

The speed of both $x$ and $y$ components are:
\[
\dot{x} = \ell \cos \theta \cdot \dot{\theta}
\qquad\qquad
\dot{y} = \ell \sin \theta \cdot \dot{\theta}
\]

Now solving for the \emph{kinetic energy}:
\begin{align*}
T = \frac{1}{2} mv^2 &= \frac{1}{2} m \left( \sqrt{\dot{x}^2 + \dot{y}^2 }\right)^2 \\
  &= \frac{1}{2} m \left( \ell^2\cos^2 \theta \cdot \dot{\theta}^2 + \ell^2\sin^2 \theta \cdot \dot{\theta}^2 \right ) \\ 
  &= \frac{1}{2} m \ell^2\dot{\theta}^2
\end{align*}

The \emph{potential energy}:
\begin{align*}
V = mgh = -mg \ell \cos \theta
\end{align*}

Now the Lagrangian becomes:
\begin{align*}
\mathcal{L} = T - V = \frac{1}{2} m \ell^2\dot{\theta}^2 + mg \ell \cos \theta
\end{align*}

There is only one degree of freedom, that is $\theta$, hence its pretty easy to solve for the equations of motion:
\begin{align*}
\frac{\partial \mathcal{L}}{\partial \dot{\theta}} &= m\ell^2 \dot{\theta} \\
\frac{d}{dt} \left( \frac{\partial \mathcal{L}}{\partial \dot{\theta}} \right) &= m\ell^2 \ddot{\theta}  \\
\frac{\partial \mathcal{L}}{\partial \theta} &= -mg \ell \sin \theta
\end{align*}

Therefore the final solution is:
\begin{align*}
\frac{d}{dt} \left( \frac{\partial \mathcal{L}}{\partial \dot{\theta}} \right) - \frac{\partial \mathcal{L}}{\partial \theta} &= 0 \\
m\ell^2 \ddot{\theta} + mg \ell \sin \theta &= 0 \\
\ddot{\theta} + \frac{g}{\ell} \sin \theta &= 0 
\end{align*}


\subsection{Solving using 4th Order Runge Kutta}

First we need to have some initial conditions:
\[
\text{At}\ t = 0 \qquad \Rightarrow \qquad \dot{\theta}(0) = 0 
\quad\text{and}\quad
\theta (0) = \theta_0
\]

To use the Runge Kutta technique, the solution needs to have first degree derivatives.  Therefore, we need to rewrite
things and use some variable substitutions to get rid of the higher order derivatives.

\begin{align*}
\intertext{Rewritting the original equation to have:}
\ddot{\theta} &= - \frac{g}{\ell} \sin \theta \\
\intertext{Substitute $\dot{u}$ for $\ddot{\theta}$}
\dot{u} &= -\frac{g}{\ell} \sin \theta \\
\intertext{The next substitution becomes:} 
\dot{u} &= \ddot{\theta} \Rightarrow u = \dot{\theta}
\end{align*}

That is:
\[
\ddot{\theta} + \frac{g}{\ell} \sin \theta = 0 
\qquad\Rightarrow\qquad
\left\{ 
\begin{array}{l}
\dot{\theta} = u \\
\dot{u} = - \frac{g}{\ell} \sin \theta
\end{array}
\right\}
\]

The $\theta$\ and $u$ values are solved with the 4th order Runge Kutta technique (which is what is used in the code):
\begin{align*}
    & \text{Solving for $\theta$}                                                   &           & \text{Solving for $u$} \\
k_1 &= \dot{\theta} (\theta_\mathsf{curr}, u_\mathsf{curr})                         &\quad  m_1 &= \dot{u}(\theta_\mathsf{curr}, u_\mathsf{curr}) \\
k_2 &= \dot{\theta} (\theta_\mathsf{curr} + 0.5 hk_1, u_\mathsf{curr} + 0.5 hk_1)   &       m_2 &= \dot{u}(\theta_\mathsf{curr} + 0.5 hm_1, u_\mathsf{curr} + 0.5 hm_1) \\
k_3 &= \dot{\theta} (\theta_\mathsf{curr} + 0.5 hk_2, u_\mathsf{curr} + 0.5 hk_2)   &       m_3 &= \dot{u}(\theta_\mathsf{curr} + 0.5 hm_2, u_\mathsf{curr} + 0.5 hm_2) \\
k_4 &= \dot{\theta} (\theta_\mathsf{curr} + hk_3, u_\mathsf{curr} + 0.5 hk_3)       &       m_4 &= \dot{u}(\theta_\mathsf{curr} + hm_3, u_\mathsf{curr} + 0.5 hm_3) \\
\theta_\mathsf{next} &= \theta_\mathsf{curr} + \frac{h}{6} (k_1 + 2k_2 + 2k_3 + k4) & u_\mathsf{next} &= u_\mathsf{curr} + \frac{h}{6} (m_1 + 2m_2 + 2m_3 + m4)
\end{align*}


\section{Damped Simple Pendulum}

The previous Lagrangian formula was missing any external forces, or it specifically assumed that they were zero.  If any
external forces are considered we get:
\[
\frac{d}{dt}\left( \frac{\partial\mathcal{L}}{\partial \dot{q}} \right) -
\frac{\partial \mathcal{L}}{\partial q} = F^\mathsf{ext} \cdot \frac{\partial r}{\partial q}
\]
If the forces do not explicitly on time; that is $\frac{\partial r_j}{\partial q} = \frac{\partial \dot{r}_j }{\partial \dot{q}} $ 
then:
\[
F^\mathsf{ext} \cdot \frac{\partial r}{\partial q} = - \frac{\partial F}{\partial \dot{q}}
\]
So with the force we have:
\[
\frac{d}{dt}\left( \frac{\partial\mathcal{L}}{\partial \dot{q}} \right) -
\frac{\partial \mathcal{L}}{\partial q} + \frac{\partial F}{\partial \dot{q}} = 0
\]

With a pendulum the linear damping (that is friction from rotating) can be written as:
\[
F = \frac{1}{2}b\dot{\theta}^2
\]

The final motion equation is:
\[
\ddot{\theta} + b\dot{\theta} + \frac{g}{\ell} \sin \theta = 0;
\]

Breaking up the equation so it can be used in the Runge Kutta technique we get:
\begin{align*}
\ddot{\theta} &= -b\dot{\theta} - \frac{g}{\ell} \sin \theta \\
\intertext{Substitute $\dot{u}$ for $\ddot{\theta}$}
\dot{u} &= -b u - \frac{g}{\ell} \sin \theta \\
\intertext{The next substitution becomes:}
\dot{u} &= \ddot{\theta} \Rightarrow u = \dot{\theta}
\end{align*}

The final equation becomes:
\[
\ddot{\theta} + b\dot{\theta} + \frac{g}{\ell} \sin \theta = 0 
\qquad\Rightarrow\qquad
\left\{ 
\begin{array}{l}
\dot{\theta} = u \\
\dot{u} = -bu - \frac{g}{\ell} \sin \theta
\end{array}
\right\}
\]





\section{Double Pendulum}

This is perhaps the simpler cases, a pendulum with damping.
\begin{center}
\psset{unit=1in}
\begin{pspicture}(0,0)(4,2)
\psframe[fillstyle=hlines,hatchcolor=gray,hatchsep=1pt](1.5,2)(2.5,1.9)
\psline(2,1.9)(2.3,0.9)(2.9,0.1)
\pscircle[fillstyle=solid,fillcolor=black](2,1.9){1.5pt}
\pscircle[fillstyle=solid,fillcolor=gray](2.3,0.9){3pt}
\pscircle[fillstyle=solid,fillcolor=gray](2.9,0.1){3pt}
\rput(2.25,1.45){$\ell_1$}
\rput(2.75,0.47){$\ell_2$}
\rput(2.47,0.9){$m_1$}
\rput(3.07,0.1){$m_2$}
\psline[linewidth=0pt](2,1.75)(2,0.95)
\psline[linewidth=0pt](2.3,0.75)(2.3,0.05)
\rput(1.88,1.05){$\theta_1$}
\psline[linewidth=0.5pt](2,1.05)(1.95,1.05)
\psarc[linewidth=0.5pt]{<->}(2,1.9){0.85}{270}{286}
\rput(2.18,0.15){$\theta_2$}
\psline[linewidth=0.5pt](2.3,0.15)(2.25,0.15)
\psarc[linewidth=0.5pt]{<->}(2.3,0.9){0.75}{270}{305.8}
\end{pspicture}
\end{center}


The positions of the masses can be calculated as:
\begin{align*}
x_1 &= \ell_1 \sin \theta_1  \\
y_1 &= -\ell_1 \cos \theta_1 \\
x_2 &= \ell_1 \sin \theta_1 + \ell_2 \sin \theta_2 \\
y_2 &= -\ell_1 \cos \theta_1 - \ell_2 \cos \theta_2
\end{align*}

The speeds of the masses can be found by taking the derivatives:
\begin{align*}
\dot{x}_1 &= \ell_1 \cos \theta_1 \cdot \dot{\theta}_1 \\
\dot{y}_1 &= \ell_1 \sin \theta_1 \cdot \dot{\theta}_1 \\
\dot{x}_2 &= \ell_1 \cos \theta_1 \cdot \dot{\theta}_1 + \ell_2 \cos \theta_2 \cdot \dot{\theta}_2 \\
\dot{y}_2 &= \ell_1 \sin \theta_1 \cdot \dot{\theta}_1 + \ell_2 \sin \theta_2 \cdot \dot{\theta}_2
\end{align*}

Now solving for the \emph{kinetic energy}:
\begin{align*}
T = \frac{1}{2} mv^2 &= \frac{1}{2} m \left( \sqrt{\dot{x}^2 + \dot{y}^2 }\right)^2 \\
  &= \frac{1}{2} m_1 \left( \dot{x}_1^2 + \dot{y}_1^2 \right) + \frac{1}{2} m_2 \left( \dot{x}_2^2 + \dot{y}_2^2 \right) \\
  &= \frac{1}{2} m_1 \left( \ell_1^2 \cos^2 \theta_1 \cdot \dot{\theta}_1^2 + \ell_1^2 \sin^2 \theta_1 \cdot \dot{\theta}_1^2 \right) \\
  &\qquad + \frac{1}{2} m_2 \left( \ell_1^2 \cos^2 \theta_1 \cdot \dot{\theta}_1^2 + 2 \ell_1\ell_2 \cos \theta_1 \cos \theta_2 \cdot \dot{\theta}_1\dot{\theta}_2 
          + \ell_2^2 \cos^2 \theta_2^2 \cdot \dot{\theta}_2^2 \right. \\
  &\qquad\qquad  +  \left.\ell_1^2 \sin^2 \theta_1 \cdot \dot{\theta}_1^2 + 2 \ell_1\ell_2 \sin \theta_1 \sin \theta_2 \cdot \dot{\theta}_1 \dot{\theta}_2 
              + \ell_2^2 \sin^2 \theta_2 \cdot \dot{\theta}_2^2 \right) \\
  &= \frac{1}{2} m_1 \ell_1^2 \dot{\theta}_1^2 + \frac{1}{2} m_2 \left( \ell_1^2 \dot{\theta}_1^2 + \ell_2^2 \dot{\theta}_2^2 
              + 2 \ell_1\ell_2 \dot{\theta}_1 \dot{\theta}_2 ( \cos \theta_1 \cos \theta_2 + \sin \theta_1 \sin \theta_2 ) \right) \\
  &= \frac{1}{2} m_1 \ell_1^2 \dot{\theta}_1^2 + \frac{1}{2} m_2 \ell_1^2 \dot{\theta}_1^2 + \frac{1}{2} m_2 \ell_2^2 \dot{\theta}_2^2 
              + m_2 \ell_1\ell_2 \dot{\theta}_1 \dot{\theta}_2 \cos (\theta_1 - \theta_2) \\
  &= \frac{1}{2} (m_1 + m_2) \ell_1^2 \dot{\theta}_1^2 + \frac{1}{2} m_2 \ell_2^2 \dot{\theta}_2^2 
              + m_2 \ell_1\ell_2 \dot{\theta}_1 \dot{\theta}_2 \cos (\theta_1 - \theta_2) \\
\end{align*}

The \emph{potential energy} is:
\begin{align*}
V = mgh &= m_1 g (-\ell_1 \cos \theta_1) + m_2 g (-\ell_1 \cos \theta_1 - \ell_2 \cos \theta_2) \\
  &= -m_1 g \ell_1 \cos \theta_1 - m_2 g \ell_1 \cos \theta_1 - m_2 g \ell_2 \cos \theta_2 \\
  &= -(m_1 + m_2)g \ell_1 \cos \theta_1 - m_2 g \ell_2 \cos \theta_2 \\
\end{align*}

Now computing the Lagrangian:
\begin{align*}
\mathcal{L} &= T - V \\
   &= \underbrace{\frac{1}{2} (m_1 + m_2) \ell_1^2 \dot{\theta}_1^2 + \frac{1}{2} m_2 \ell_2^2 \dot{\theta}_2^2
          + m_2 \ell_1\ell_2 \dot{\theta}_1 \dot{\theta}_2 \cos (\theta_1 - \theta_2)}_T
      + \underbrace{\vphantom{\frac{1}{2}}(m_1 + m_2)g \ell_1 \cos \theta_1 + m_2 g \ell_2 \cos \theta_2}_{-V} \\
\end{align*}


There are a few degrees of freedom, that is $\theta_1$ and $\theta_2$.  So to determine the equations of motions, we need
to compute:
\[
\frac{d}{dt}\left( \frac{\partial\mathcal{L}}{\partial \dot{\theta}_1} \right) - 
\frac{\partial\mathcal{L}}{\partial \theta_1} = 0 
\qquad\qquad
\text{and}
\qquad\qquad
\frac{d}{dt}\left( \frac{\partial\mathcal{L}}{\partial \dot{\theta}_2} \right) - 
\frac{\partial\mathcal{L}}{\partial \theta_2} = 0 
\] 

\begin{align*}
\frac{\partial \mathcal{L}}{\partial \dot{\theta}_1} &= (m_1 + m_2)\ell_1^2 \dot{\theta}_1 + m_2 \ell_1 \ell_2 \cos (\theta_1 - \theta_2) \dot{\theta}_2 \\
\frac{d}{dt} \left( \frac{\partial \mathcal{L}}{\partial \dot{\theta}_1} \right) &= 
     (m_1 + m_2)\ell_1^2 \ddot{\theta}_1 + m_2\ell_1\ell_2 \cos (\theta_1 - \theta_2) \ddot{\theta}_2 
           - m_2\ell_1\ell_2 \dot{\theta}_2 \sin (\theta_1 - \theta_2) \cdot (\dot{\theta}_1 - \dot{\theta}_2) \\
 &\ \text{(Note: remember to chain the angle!)} \\
\frac{\partial \mathcal{L}}{\partial \theta_1} &= - m_2 \ell_1\ell_2\sin (\theta_1 - \theta_2) \dot{\theta}_1 \dot{\theta}_2 - (m_1 + m_2) g \ell_1 \sin \theta_1 \\
\frac{d}{dt} \left( \frac{\partial \mathcal{L}}{\partial \dot{\theta}_1} \right) - \frac{\partial \mathcal{L}}{\partial \theta_1} &=
     (m_1 + m_2)\ell_1^2 \ddot{\theta}_1 + m_2 \ell_1 \ell_2 \cos (\theta_1 - \theta_2) \ddot{\theta}_2 
          - m_2\ell_1\ell_2 \dot{\theta}_2 \sin (\theta_1 - \theta_2) \cdot (\dot{\theta}_1 - \dot{\theta}_2)  \\
     &\qquad + m_2 \ell_1\ell_2 \sin (\theta_1 - \theta_2)\dot{\theta}_1 \dot{\theta}_2  + (m_1 + m_2) g \ell_1 \sin \theta_1  \\
     &= (m_1 + m_2)\ell_1^2 \ddot{\theta}_1 + m_2 \ell_1 \ell_2 \cos (\theta_1 - \theta_2) \ddot{\theta}_2
          + m_2\ell_1\ell_2 \sin (\theta_1 - \theta_2) \dot{\theta}_2^2 + (m_1 + m_2) g \ell_1 \sin \theta_1  \\
     &= (m_1 + m_2)\ell_1 \ddot{\theta}_1 + m_2 \ell_2 \cos (\theta_1 - \theta_2) \ddot{\theta}_2
          + m_2\ell_2 \sin (\theta_1 - \theta_2) \dot{\theta}_2^2 + (m_1 + m_2) g \sin \theta_1  \\
\end{align*}

\begin{align*}
\frac{\partial \mathcal{L}}{\partial \dot{\theta}_2} &= m_2 \ell_2^2 \dot{\theta}_2 + m_2\ell_1\ell_2 \cos(\theta_1 - \theta_2) \dot{\theta}_1 \\
\frac{d}{dt} \left( \frac{\partial \mathcal{L}}{\partial \dot{\theta}_2} \right) &= m_2 \ell_2^2 \ddot{\theta}_2 
    + m_2\ell_1\ell_2 \cos(\theta_1 - \theta_2) \ddot{\theta}_1 - m_2\ell_1\ell_2 \sin (\theta_1 - \theta_2) \dot{\theta}_1 \cdot (\dot{\theta}_1 - \dot{\theta}_2) \\
\frac{\partial \mathcal{L}}{\partial \theta_2} &= m_2 \ell_1\ell_2 \dot{\theta}_1 \dot{\theta}_2 \sin (\theta_1 - \theta_2) - m_2 g \ell_2 \sin \theta_2 \\
\frac{d}{dt} \left( \frac{\partial \mathcal{L}}{\partial \dot{\theta}_2} \right) - \frac{\partial \mathcal{L}}{\partial \theta_2} &= 
       m_2 \ell_2^2 \ddot{\theta}_2 + m_2\ell_1\ell_2 \cos(\theta_1 - \theta_2) \ddot{\theta}_1 - m_2\ell_1\ell_2 \sin (\theta_1 - \theta_2) \dot{\theta}_1 \cdot (\dot{\theta}_1 - \dot{\theta}_2) \\
  &\qquad - m_2 \ell_1\ell_2 \dot{\theta}_1 \dot{\theta}_2 \sin (\theta_1 - \theta_2) + m_2 g \ell_2 \sin \theta_2 \\
  &= m_2 \ell_2^2 \ddot{\theta}_2 + m_2\ell_1\ell_2 \cos(\theta_1 - \theta_2) \ddot{\theta}_1 - m_2 \ell_1\ell_2 \sin (\theta_1 - \theta_2) \dot{\theta}_1^2 
       + m_2 g \ell_2 \sin \theta_2 \\
  &= \ell_2 \ddot{\theta}_2 + \ell_1 \cos(\theta_1 - \theta_2) \ddot{\theta}_1 - \ell_1 \sin (\theta_1 - \theta_2) \dot{\theta}_1^2 + g \sin \theta_2 \\
\end{align*}

Therefore, the equations of motion (or governing equations) are:
\begin{align*}
(m_1 + m_2)\ell_1 \ddot{\theta}_1 + m_2 \ell_2 \cos (\theta_1 - \theta_2) \ddot{\theta}_2
          + m_2\ell_2 \sin (\theta_1 - \theta_2) \dot{\theta}_2^2 + (m_1 + m_2) g \sin \theta_1 &= 0 \\
  \ell_2 \ddot{\theta}_2 + \ell_1 \cos(\theta_1 - \theta_2) \ddot{\theta}_1 - \ell_1 \sin (\theta_1 - \theta_2) \dot{\theta}_1^2 + g \sin \theta_2 &= 0 
\end{align*}

Note that if $m_2 = 0$, $\ell_2 = 0$, and $\theta_1 = \theta_2$, then the equations simplify to a simple pendulum.

\subsection{Solving Numerically for the Runge Kutta technique}

To be able to solve numerically, the equations of motion need to be rewritten in the form:
\[
\ddot{\theta}_1 = f (\theta_1, \theta_2, \dot{\theta}_1, \dot{\theta}_2)
\qquad\qquad
\ddot{\theta}_2 = f (\theta_1, \theta_2, \dot{\theta}_1, \dot{\theta}_2)
\]

First rearranging the equation of motions:
\begin{align*}
(m_1 + m_2)\ell_1 \ddot{\theta}_1 + m_2 \ell_2 \cos (\theta_1 - \theta_2) \ddot{\theta}_2 &=
          - m_2\ell_2 \sin (\theta_1 - \theta_2) \dot{\theta}_2^2 - (m_1 + m_2) g \sin \theta_1  \\
\ell_1 \cos(\theta_1 - \theta_2) \ddot{\theta}_1 + \ell_2 \ddot{\theta}_2 &=
           \ell_1 \sin (\theta_1 - \theta_2) \dot{\theta}_1^2 - g \sin \theta_2 
\end{align*}

Rewrite the equations using some matrices:
\[
\left[
\begin{array}{cc}
(m_1 + m_2)\ell_1 & m_2 \ell_2 \cos (\theta_1 - \theta_2) \\
\ell_1 \cos(\theta_1 - \theta_2) & \ell_2 
\end{array}
\right]
\left[
\begin{array}{c}
\ddot{\theta}_1 \\
\ddot{\theta}_2
\end{array}
\right]
= 
\left[
\begin{array}{c}
- m_2\ell_2 \sin (\theta_1 - \theta_2) \dot{\theta}_2^2 - (m_1 + m_2) g \sin \theta_1 \\
\ell_1 \sin (\theta_1 - \theta_2) \dot{\theta}_1^2 - g \sin \theta_2
\end{array}
\right]
\]

Now we have something in the form:
\[
[M] \ddot{\vec{\theta}} = \vec{F}
\qquad \text{or} \qquad
\left[
\begin{array}{cc}
m_{11} & m_{12} \\
m_{21} & m_{22}
\end{array}
\right]
\left[
\begin{array}{c}
\ddot{\theta}_1 \\
\ddot{\theta}_2
\end{array}
\right]
=
\left[
\begin{array}{c}
F_1 \\
F_2
\end{array}
\right]
\]
So to find $\ddot{\theta}_1$\ and $\ddot{\theta}_2$, we just have to just take the inverse of the matrix:
\[
\left[
\begin{array}{c}
\ddot{\theta}_1 \\
\ddot{\theta}_2
\end{array}
\right]
=
[M]^{-1} 
\left[
\begin{array}{c}
F_1 \\
F_2
\end{array}
\right]
\qquad
\text{Where:}
\quad
[M]^{-1} = 
\frac{1}{m_{11}m_{22} - m_{12}m_{21}}
\left[
\begin{array}{cc}
 m_{22} & -m_{21} \\
-m_{12} &  m_{11}
\end{array}
\right]
\]

Instead of figuring out the determinant, the inverse, and then the solution, it will be all done numerically
(that is let the computer do the heavy lifting).

\subsection{Damping}

Damping can be introduced to the lagrangian to make the results appear to be slightly more
realistic.  However it should be noted that it is a very rough approximation.  In the Lagrangian
we would have:
\[
\frac{d}{dt} \left( \frac{\partial \mathcal{L}}{\partial \dot{\theta}_i} \right) - \frac{\partial \mathcal{L}}{\partial \theta_i} 
= \frac{1}{2} b_i\dot{\theta}_i^2
\]
Where $b_i$ is a damping constant.  To keep our model a little more simple, all $b_i$'s are the same. 


\section{Spring Pendulum}

\begin{center}
\psset{unit=1in}
\begin{pspicture}(0,0.25)(4,2)
\psframe[fillstyle=hlines,hatchcolor=gray,hatchsep=1pt](1.5,2)(2.5,1.9)
\pscoil[coilwidth=0.25cm](2,1.9)(3,0.5)
\pscircle[fillstyle=solid,fillcolor=black](2,1.9){1.5pt}
\pscircle[fillstyle=solid,fillcolor=gray](3,0.5){3pt}
\rput[l](2.65,1.3){$\ell(t) = \ell_0 + x(t)$}
\rput(3.15,0.5){$m$}
\psline[linewidth=0pt](2,1.75)(2,0.75)
\rput(2.35,0.85){$\theta$}
\psarc[linewidth=0.5pt]{<-}(2,1.9){1.1}{270}{285}
\psarc[linewidth=0.5pt]{->}(2,1.9){1.1}{292}{303}
\end{pspicture}
\end{center}

There are two degrees of freedom in this problem, which are taken to be the angle of the pendulum from the vertical $\theta$ and 
the total length of the spring.  The spring is assumed to have a force constant $k$ and an equilibrium length $\ell_0$.

The position of the mass due to the motion of the pendulum and spring is:
\[
x = \ell \sin \theta = (\ell_0 + x(t)) \sin \theta 
\qquad\qquad
y = -\ell \cos \theta = -(\ell_0 + x(t)) \cos \theta
\]

The velocity is a bit more complicated (remember to take both derivatives):
\[
\dot{x} = (\ell_0 + x(t)) \cos \theta \cdot \dot{\theta} + \dot{x} \cdot \sin \theta
\qquad\qquad
\dot{y} = (\ell_0 + x(t)) \sin \theta \cdot \dot{\theta} - \dot{x} \cdot \cos \theta
\]

The \emph{kinetic energy} is:
\begin{align*}
T = \frac{1}{2}mv^2 &= \frac{1}{2}m \left( v_\mathsf{spring}^2 + v_\mathsf{pend}^2 \right) \\
      &= \frac{1}{2}m \left( \dot{x}^2 + \dot{y}^2 \right) \\
      &= \frac{1}{2}m \left( ((\ell_0 + x(t)) \cos \theta \cdot \dot{\theta} + \dot{x} \cdot \sin \theta)^2 + ((\ell_0 + x(t)) \sin \theta\cdot\dot{\theta} - \dot{x} \cdot \cos \theta)^2 \right) \\
      &= \frac{1}{2}m \left( (\ell_0 + x)^2 \cos^2 \theta \dot{\theta}^2 + 2(\ell_0 + x)\cos \theta\sin \theta \dot{\theta} \dot{x}+ \sin^2 \theta\dot{x}^2 \right) + \\ 
      & \qquad\qquad\qquad \left( (\ell_0 + x)^2 \sin^2 \theta \dot{\theta}^2 - 2(\ell_0 + x)\sin \theta\cos \theta \dot{\theta} \dot{x} + \cos^2 \theta \dot{x}^2 \right) \\
      &= \frac{1}{2}m \left( (\ell_0 + x)^2 \dot{\theta}^2 + \dot{x}^2 \right) \\
\end{align*}

The \emph{potential energy} is:
\begin{align*}
V = mgh + \frac{1}{2} kx^2 
  &= -mg\ell\cos\theta + \frac{1}{2} kx^2 \\
  &= -mg(\ell_0 + x)\cos\theta + \frac{1}{2} kx^2
\end{align*}

The \emph{Lagrangian} now becomes:
\begin{align*}
\mathcal{L} &= T - V \\
  &= \frac{1}{2} m \dot{x}^2 + \frac{1}{2}m (\ell_0 + x)^2 \dot{\theta}^2 + mg(\ell_0 + x)\cos\theta - \frac{1}{2} kx^2
\end{align*}

To find the equations of motion the Lagrangian needs to be solved twice:
\begin{align*}
\frac{\partial \mathcal{L}}{\partial \dot{\theta}} &= m (\ell_0 + x)^2 \dot{\theta} \\
\frac{d}{dt} \left( \frac{\partial \mathcal{L}}{\partial \dot{\theta}} \right) &=  m (\ell_0 + x)^2 \ddot{\theta} + 2m (\ell_0 + x) \dot{\theta} \dot{x}\\
\frac{\partial \mathcal{L}}{\partial \theta} &= -mg(\ell_0 + x)\sin\theta \\
\frac{d}{dt} \left( \frac{\partial \mathcal{L}}{\partial \dot{\theta}} \right) - \frac{\partial \mathcal{L}}{\partial \theta} &=
 m (\ell_0 + x)^2 \ddot{\theta} + 2m (\ell_0 + x) \dot{\theta} \dot{x} + mg(\ell_0 + x)\sin\theta = 0 \\
&= (\ell_0 + x)^2 \ddot{\theta} + 2 (\ell_0 + x) \dot{\theta} \dot{x} + g(\ell_0 + x)\sin\theta = 0
\end{align*}

\begin{align*}
\frac{\partial \mathcal{L}}{\partial \dot{x}} &= m \dot{x} \\
\frac{d}{dt} \left( \frac{\partial \mathcal{L}}{\partial \dot{x}} \right) &= m \ddot{x} \\
\frac{\partial \mathcal{L}}{\partial x} &= m (\ell_0 x) \dot{\theta}^2 + mg \cos \theta - kx \\
\frac{d}{dt} \left( \frac{\partial \mathcal{L}}{\partial \dot{x}} \right) - \frac{\partial \mathcal{L}}{\partial x} &=
m \ddot{x} - m (\ell_0 + x) \dot{\theta}^2 - mg \cos \theta + kx = 0 \\
&= \ddot{x} - (\ell_0 + x) \dot{\theta}^2 - g \cos \theta + \frac{k}{m}x = 0
\end{align*}

The final equations of motion are:
\begin{align*}
\ddot{\theta} &= - \frac{2}{(\ell_0 + x)} \dot{\theta} \dot{x} - \frac{g}{(\ell_0 + x)} \sin\theta \\
\ddot{x} &= (\ell_0 + x) \dot{\theta}^2 + g \cos \theta - \frac{k}{m}x
\end{align*}



\subsection{Damping}

Adding some damping can make the simulation slightly more realistic (that is so the motion eventually slows down and stops).   However, 
to keep things simple a very easy to calculate form of damping was added, but it isn't too accurate.  So we have:

\[
\frac{d}{dt} \left( \frac{\partial \mathcal{L}}{\partial \dot{q}} \right) - \frac{\partial \mathcal{L}}{\partial q} 
= \frac{\partial F}{\partial \dot{q}}
\]
Again to keep things simple:
\[
F_{\dot{\theta}} = \frac{1}{2} b\dot{\theta}^2 
\qquad\qquad
F_{\dot{x}} = \frac{1}{2} c \dot{x}^2
\]
So for each simulation step we subtract the following from $\ddot{\theta}$ and $\ddot{x}$:
\[
\frac{\partial F}{\partial \dot{\theta}} = b\dot{\theta}
\qquad\qquad
\frac{\partial F}{\partial \dot{x}} = c\dot{x}
\]

\end{document}




